\documentclass[10pt]{article}
\usepackage{amsmath,amssymb}
\usepackage{tikz}
\usepackage{fancyhdr}

\usepackage[margin=1in,tmargin=1.25in,paper=letterpaper]{geometry}

\setlength{\headheight}{0pt}
\pagestyle{fancy}
\lhead{AP Physics C}
\chead{Unit 9 Practice Test}
\rhead{Forest Kobayashi}

\usepackage[inline]{enumitem}
\usepackage{booktabs}

\renewcommand{\labelenumii}{\Alph{enumii})}
\setlist[enumerate]{itemsep=0mm}
\usepackage{siunitx}

\begin{document}

\begin{enumerate}[itemsep=5mm]
  \item
    The electric field $\vec{E}$ just outside the surface of a
    charged conductor is
    \begin{enumerate}
      \item directed perpendicular to the surface
      \item directed parallel to the surface
      \item independent of the surface charge density $\sigma$
      \item zero
      \item infinite
    \end{enumerate}

  \item
    A point charge is placed at the center of an uncharged,
    spherical, conducting shell of radius $R$.  The electric field
    inside and outside the sphere are measured.  The point charge is
    then moved off center a distance $R/2$ and the fields are measured
    again.  What is the effect on the fields?
    \begin{enumerate}
      \item Changed neither outside nor inside
      \item Changed inside but not changed outside
      \item Not changed inside but changed outside
      \item Changed inside and outside
    \end{enumerate}

  \item
    The net electric flux through a closed surface is
    \begin{enumerate}
      \item infinite only if there are no charges enclosed by the
        surface
      \item infinite onyl if the net charge enclosed by the surface is
        zero
      \item zero if only negative charges are enclosed by the surface
      \item zero if only positive charges are enclosed by the surface
      \item zero if the net charge enclosed by the surface is zero
    \end{enumerate}

  \item
    Two metal spheres that are initially uncharged are mounted on
    insulating stands, as shown to the right.  A negatively charged
    rubber rod is brought close to, but does not make contact with,
    sphere $X$.  Sphere $Y$ is then brought close to $X$ on the side
    opposite the rubber rod.  $Y$ is allowed to touch $X$ and then is
    removed some distance away.  The rubber rod is then moved far away
    from $X$ and $Y$.  What are the final charges on the spheres?
    \begin{table}[h]
      \hspace{1cm}
      \begin{tabular}{@{}lll@{}}
        \cmidrule(l){2-3}
           & Sphere $X$ & Sphere $Y$ \\ \cmidrule(l){2-3}
        A) & Zero       & Zero       \\
        B) & Negative   & Negative   \\
        C) & Negative   & Positive   \\
        D) & Positive   & Negative   \\
        E) & Positive   & Positive   \\ \cmidrule(l){2-3}
      \end{tabular}
    \end{table}
\end{enumerate}
{\it For questions 5 and 6:} The point charge $Q$ shown to the right
  is at the center of a cube
\begin{enumerate}[resume, itemsep = 5mm]
  \item
    If the cube is a Gaussian surface, then the electric flux
    through any one of the faces of the cube is
    \begin{enumerate*}
      \item $\displaystyle \frac{Q}{\varepsilon_0}$
      \item $\displaystyle \frac{Q}{4\pi\varepsilon_0}$
      \item $\displaystyle \frac{Q}{3\varepsilon_0}$
      \item $\displaystyle \frac{Q}{6\varepsilon_0}$
      \item $\displaystyle \frac{Q}{12\varepsilon_0}$
    \end{enumerate*}
  \item If the cube is a metal box that is isolated, ungrounded, and
    uncharged, which of the following is true?
    \begin{enumerate}
      \item The net charge on the outside surface of the box is $Q$
      \item The electric field inside the box is zero.
      \item The electric field inside the box is uniform, but not
        zero.
      \item The electric field outside the box is zero everywhere.
      \item The electric field outside the box is the same as if only
        the point charge (and not the box) were there.
    \end{enumerate}
  \item A small object has charge $Q$.  A charge $q$ is removed from
    it and placed on another small object.  The two objects are held a
    certain distance apart.  For the repulsive force between the two
    objects to be a maximum, $q$ should be
    \begin{enumerate}
      \item 2$Q$
      \item $Q$
      \item $Q/2$
      \item $Q/4$
      \item 0\si{\coulomb}
    \end{enumerate}
\end{enumerate}

\noindent {\it For questions 8 and 9}: A positively charged insulating rod is
brought close to an object that is suspended by an insulating thread.

\begin{enumerate}[resume, itemsep = 5mm]
  \item
    If the object is attracted to the rod we can conclude
    \begin{enumerate}
      \item that the object is positively charged
      \item that the object is negatively charged
      \item that the object is an insulator
      \item that the object is a conductor
      \item none of the above
    \end{enumerate}

  \item
    If the object is repelled away from the rod we can conclude
    \begin{enumerate}
      \item that the object is positively charged
      \item that the object is negatively charged
      \item that the object is an insulator
      \item that the object is a conductor
      \item none of the above
    \end{enumerate}

  \item
    Two point charges have equal and opposite magnitudes $+Q$ and
    $-Q$.  Aside from infinite distances, the electric field is zero
    \begin{enumerate}
      \item midway between $+Q$ and $-Q$
      \item on the perpendicular biector of the line joining $+Q$ and
        $-Q$, but not on the line itself.
      \item on the line joining $+Q$ and $-Q$, to the side of $Q$
        opposite $-Q$
      \item on the line joining $+Q$ and $-Q$, to the side of $-Q$
        opposite $Q$
      \item at none of these places (there is no point at finite
        distances)
    \end{enumerate}

  \item The electric field due to a charge uniformly distributed over
    the surface of a spherical shell is zero
    \begin{enumerate}
      \item everywhere
      \item nowhere
      \item only at the center of the shell
      \item only inside the shell
      \item only outside the shell
    \end{enumerate}

  \item A charge $Q$ is distributed uniformly throughout an insulating
    sphere of radius $R$.  The magnitude of the electric field at a
    point $R/2$ from the center of the sphere is
    \begin{enumerate}
      \item $\displaystyle \frac{Q}{4\pi\varepsilon_0R^2}$
      \item $\displaystyle \frac{Q}{\pi\varepsilon_0R^2}$
      \item $\displaystyle \frac{3Q}{4\pi\varepsilon_0R^2}$
      \item $\displaystyle \frac{Q}{8\pi\varepsilon_0R^2}$
      \item none of these
    \end{enumerate}

  \item Charge is distributed uniformly along a very long straight
    wire.  The electric field $2\si{\centi\meter}$ from the wire has
    magnitude $20\si{\newton\per\coulomb}$.  The magnitude of the
    electric field $4\si{\centi\meter}$ from the wire is
    \begin{enumerate}
      \item $120\si{\newton\per\coulomb}$
      \item $80\si{\newton\per\coulomb}$
      \item $40\si{\newton\per\coulomb}$
      \item $10\si{\newton\per\coulomb}$
      \item $5\si{\newton\per\coulomb}$
    \end{enumerate}

  \item A long line of charge with charge per unit length
    $\lambda_\ell$ runs along the axis of a cylindrical conducting
    shell that carries a charge per unit length $\lambda_c$.  The
    charge per unit length on the inner and outer surfaces of the
    shell are:
    \begin{enumerate}
      \item $\lambda_\ell$ and $\lambda_c$
      \item $-\lambda_\ell$ and $\lambda_c + \lambda_\ell$
      \item $-\lambda_\ell$ and $\lambda_c - \lambda_\ell$
      \item $\lambda_c + \lambda_\ell$ and $\lambda_c - \lambda_\ell$
      \item $\lambda_\ell - \lambda_c$ and $\lambda_c + \lambda_\ell$
    \end{enumerate}

  \item Two very large parallel conducting paltes carry charge of
    equal magnitude, distributed uniformly over their inner surfaces
    as shown to the right.  The ranking of the magnitude of the
    electric field at points 1 - 5 is
    \begin{enumerate}
      \item $1 < 2 < 3 < 4 < 5$
      \item $5 < 4 < 3 < 2 < 1$
      \item $1 = 4 = 5 < 2 = 3$
      \item $2 = 3 < 1 = 4 < 5$
      \item $2 = 3 < 1 = 4 = 5$
    \end{enumerate}

\end{enumerate}

\end{document}
